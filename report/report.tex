\documentclass[12pt]{article}
\usepackage[utf8]{inputenc}
\usepackage{polski}
\usepackage[polish]{babel}
\usepackage{graphicx}
\usepackage{listings}
\usepackage{amsmath}
\usepackage{hyperref}
\usepackage{booktabs}
\usepackage{rotating}
\usepackage{pgfplots}
\usepackage{pgfplotstable}
\usepackage{adjustbox}
\usepackage{seqsplit}
\pgfplotsset{compat=1.18}

\title{Implementacja i analiza indeksowo-sekwencyjnej organizacji plików}
\author{Wojciech Trapkowski}
\date{\today}

\begin{document}
\maketitle

\section{Wprowadzenie}
Celem projektu jest implementacja i analiza indeksowo-sekwencyjnej organizacji plików (ISAM - Indexed Sequential Access Method), która łączy zalety dostępu sekwencyjnego i bezpośredniego do danych. Metoda ta została opracowana przez IBM w latach 60-tych i do dziś stanowi podstawę wielu systemów bazodanowych.

Podstawowe założenia tej organizacji plików obejmują:
\begin{itemize}
    \item Przechowywanie danych w uporządkowanej sekwencji rekordów, pogrupowanych w strony o stałym rozmiarze
    \item Utrzymywanie oddzielnego pliku indeksowego, zawierającego klucze i wskaźniki do odpowiadających im stron w pliku głównym
    \item Wykorzystanie obszaru przepełnień do obsługi nowych rekordów, których nie można umieścić w pierwotnie przydzielonych stronach
    \item Okresową reorganizację pliku w celu optymalizacji jego struktury
\end{itemize}

Organizacja ISAM zapewnia wydajne operacje wyszukiwania dzięki indeksom, zachowując jednocześnie możliwość sekwencyjnego przetwarzania danych. Jest szczególnie efektywna w systemach, gdzie stosunek operacji odczytu do zapisu jest wysoki, a dane są względnie statyczne.

\section{Struktury plików}
\begin{itemize}    
    \item Struktura pliku indeksowego zawiera strony indeksowe, gdzie każdy wpis składa się z klucza początkowego oraz wskaźnika do odpowiedniej strony w pliku głównym.

    \item Struktura pliku głównego składa się z nagłówka zawierającego liczbę stron oraz wskaźnik do obszaru przepełnień (strażnika), a następnie sekwencji stron zawierających rekordy.
    
    \item Obszar przepełnień służy do przechowywania rekordów, które nie mogą być umieszczone w pierwotnie przydzielonych stronach. Każdy rekord w obszarze głównym może wskazywać na dodatkowe rekordy w obszarze przepełnień.
    
    \item Organizacja rekordów w stronach opiera się na strukturze Page, która zawiera stałą liczbę wpisów. Każdy wpis zawiera klucz, wartość (PESEL), wskaźnik do obszaru przepełnień oraz flagę usunięcia.
\end{itemize}

\section{Szczegóły implementacyjne}
\subsection{Buforowanie w pamięci operacyjnej}
\begin{itemize}
    \item Mechanizm buforowania zaimplementowany jest w klasie PageBuffer, która wykorzystuje inteligentne wskaźniki do zarządzania stronami w pamięci.
    
    \item Wielkość bufora jest określona przez stałą, która definiuje maksymalną liczbę stron przechowywanych jednocześnie w pamięci.
    
    \item Strategia zastępowania stron opiera się na liczbie referencji do strony - usuwane są strony z pojedynczą referencją. Przed usunięciem strony z bufora, jej zawartość jest zapisywana na dysk.
    
    \item System śledzi liczbę operacji odczytu i zapisu poprzez liczniki.
\end{itemize}

\subsection{Parametry implementacyjne}
\begin{itemize}
    \item Rozmiar strony (PAGE\_SIZE) jest stałą określającą liczbę rekordów w pojedynczej stronie.
    
    \item Współczynnik wypełnienia ($\alpha$) określa maksymalną liczbę rekordów w stronie po reorganizacji.
    
    \item Współczynnik obszaru przepełnień ($\beta$) definiuje stosunek rozmiaru obszaru przepełnień do obszaru głównego
    
    \item Reorganizacja jest wykonywana gdy liczba rekordów w obszarze przepełnień przekroczy ustalony próg ($\gamma$).
\end{itemize}

\section{Format pliku testowego}
\subsection{Struktura rekordu}
W implementacji rekord jest reprezentowany jako pojedyncza liczba całkowita typu uint64\_t, przechowująca numer PESEL.

\section{Prezentacja wyników}
\subsection{Interfejs użytkownika}
Program oferuje interaktywny interfejs wiersza poleceń oraz możliwość wykonywania komend z pliku. Dostępne są następujące tryby pracy:
\begin{itemize}
    \item Tryb interaktywny - oznaczony znakiem zachęty >
    \item Tryb wsadowy - wykonywanie komend z pliku
\end{itemize}

\subsection{Dostępne komendy}
Program obsługuje następujące polecenia:
\begin{itemize}
    \item \texttt{insert <klucz> <wartość>} - wstawia nowy rekord
    \item \texttt{update <klucz> <wartość>} - aktualizuje istniejący rekord
    \item \texttt{search <klucz>} - wyszukuje rekord o podanym kluczu
    \item \texttt{remove <klucz>} - usuwa rekord o podanym kluczu
    \item \texttt{print} - wyświetla zawartość całej bazy danych
    \item \texttt{print\_stats} - wyświetla statystyki (liczba operacji I/O)
    \item \texttt{generate <liczba\_kluczy>} - generuje zadaną liczbę losowych rekordów
    \item \texttt{reorganise} - wymusza reorganizację struktury
    \item \texttt{flush} - wymusza zapis buforowanych danych na dysk
    \item \texttt{help} - wyświetla listę dostępnych komend
    \item \texttt{exit}/\texttt{quit} - kończy działanie programu
\end{itemize}

\subsection{Format wyświetlania}
\begin{itemize}
    \item Wyniki wyszukiwania są wyświetlane w formacie: wartość rekordu lub komunikat "Not found" dla nieznalezionych kluczy
    \item Błędy operacji są sygnalizowane odpowiednimi komunikatami
    \item Statystyki pokazują liczbę operacji odczytu i zapisu wykonanych na dysku
\end{itemize}

\section{Eksperymenty}
\subsection{Metodologia}
\begin{itemize}
    \item Przeprowadzone testy obejmowały analizę wydajności programu poprzez żonglowanie wszystkimi kluczowymi parametrami:
    \begin{itemize}
        \item Rozmiar strony (współczynnik blokowania pliku)
        \item Rozmiar bufora przechowującego strony w pamięci operacyjnej
        \item Liczba rekordów w bazie danych
        \item Współczynnik wypełnienia strony ($\alpha$)
        \item Współczynnik rozmiaru obszaru przepełnień ($\beta$)
        \item Próg reorganizacji ($\gamma$)
    \end{itemize}

    \item Mierzone metryki obejmowały:
    \begin{itemize}
        \item Liczba operacji dyskowych przy:
        \begin{itemize}
            \item Usuwaniu rekordów
            \item Aktualizacji rekordów
            \item Wyszukiwaniu rekordów
            \item Wstawianiu nowych rekordów
            \item Reorganizacji struktury
        \end{itemize}
        \item Zużycie pamięci przez pliki bazy danych
    \end{itemize}

    \item Metodyka pomiarów:
    \begin{itemize}
        \item Każda operacja była wykonywana na świeżo zainicjalizowanej bazie danych
        \item Liczniki operacji I/O były zerowane przed każdą operacją
        \item Pomiary były agregowane dla serii identycznych operacji w celu uzyskania średnich wartości
        \item Testy przeprowadzano dla różnych kombinacji parametrów, aby zbadać ich wzajemny wpływ
    \end{itemize}
\end{itemize}


\subsection{Wyniki}
Tabela 1 przedstawia wyniki przeprowadzonych testów wydajnościowych. Poszczególne kolumny reprezentują:

\begin{itemize}
    \item Kolumny operacji dyskowych (I/O):
    \begin{itemize}
        \item A - Liczba operacji dyskowych przy usuwaniu rekordów
        \item B - Liczba operacji dyskowych przy aktualizacji istniejących rekordów
        \item C - Liczba operacji dyskowych przy wyszukiwaniu rekordów
        \item D - Liczba operacji dyskowych przy wstawianiu nowych rekordów
        \item E - Liczba operacji dyskowych podczas reorganizacji struktury
    \end{itemize}
    
    \item Parametry pamięciowe:
    \begin{itemize}
        \item F - Całkowita pamięć zajmowana przez pliki bazy danych (KB)
        \item G - Współczynnik blokowania pliku (rozmiar strony)
        \item H - Rozmiar bufora przechowującego strony w pamięci operacyjnej
    \end{itemize}
    
    \item Parametry konfiguracyjne:
    \begin{itemize}
        \item I - Liczba rekordów w bazie danych
        \item J - Współczynnik wypełnienia strony ($\alpha$)
        \item K - Współczynnik rozmiaru obszaru przepełnień ($\beta$)
        \item L - Próg reorganizacji ($\gamma$)
    \end{itemize}
\end{itemize}

\begin{figure}[h]
\begin{adjustbox}{width=1.25\textwidth, center}
\pgfplotstabletypeset[
    col sep=semicolon,
    every head row/.style={
        before row=\hline,
        after row=\hline
    },
    every last row/.style={after row=\hline},
    every row/.style={before row=\vspace{-2pt}},
    font=\footnotesize,
    every column/.style={
        column type=p{1.2cm},
        text width=1.2cm,
        align=center
    },
    columns/0/.style={int detect, column name={\begin{sideways}\footnotesize\hspace{0.3cm}Liczba op.\\ dyskowych\\ przy\\ usuwaniu\hspace{0.3cm}\end{sideways}}},
    columns/1/.style={int detect, column name={\begin{sideways}\footnotesize\hspace{0.3cm}Liczba op.\\ dyskowych\\ przy\\ aktualizacji\hspace{0.3cm}\end{sideways}}},
    columns/2/.style={int detect, column name={\begin{sideways}\footnotesize\hspace{0.3cm}Liczba op.\\ dyskowych\\ przy\\ wyszukiwaniu\hspace{0.3cm}\end{sideways}}},
    columns/3/.style={int detect, column name={\begin{sideways}\footnotesize\hspace{0.3cm}Liczba op.\\ dyskowych\\ przy\\ wstawianiu\hspace{0.3cm}\end{sideways}}},
    columns/4/.style={int detect, column name={\begin{sideways}\footnotesize\hspace{0.3cm}Liczba op.\\ dyskowych\\ przy\\ reorganizacji\hspace{0.3cm}\end{sideways}}},
    columns/5/.style={int detect, column name={\begin{sideways}\footnotesize\hspace{0.3cm}Pamięć\\ (KB)\hspace{0.3cm}\end{sideways}}},
    columns/6/.style={int detect, column name={\begin{sideways}\footnotesize\hspace{0.3cm}Rozmiar\\ strony\hspace{0.3cm}\end{sideways}}},
    columns/7/.style={int detect, column name={\begin{sideways}\footnotesize\hspace{0.3cm}Rozmiar\\ bufora\hspace{0.3cm}\end{sideways}}},
    columns/8/.style={int detect, column name={\begin{sideways}\footnotesize\hspace{0.3cm}Liczba\\ rekordów\hspace{0.3cm}\end{sideways}}},
    columns/9/.style={
        column name={\begin{sideways}\footnotesize\hspace{0.3cm}$\alpha$\hspace{0.3cm}\end{sideways}},
        string replace*={0comma50}{0,50},
        string replace*={0comma75}{0,75}
    },
    columns/10/.style={
        column name={\begin{sideways}\footnotesize\hspace{0.3cm}$\beta$\hspace{0.3cm}\end{sideways}},
        string replace*={0comma50}{0,50},
        string replace*={0comma75}{0,75}
    }
]{data.csv}
\end{adjustbox} 
\end{figure}

\section{Wnioski}
\begin{itemize}
    \item Wpływ rozmiaru strony:
    \begin{itemize}
        \item Większy rozmiar strony znacząco redukuje liczbę operacji I/O
        \item Jednak zbyt duży rozmiar strony może prowadzić do nieefektywnego wykorzystania pamięci
        \item Optymalny rozmiar strony zależy od charakterystyki danych i częstotliwości operacji
    \end{itemize}

    \item Wpływ rozmiaru bufora:
    \begin{itemize}
        \item Większy bufor stron znacząco zmniejsza liczbę fizycznych operacji I/O
        \item Szczególnie efektywny przy operacjach wyszukiwania i aktualizacji
        \item Należy znaleźć kompromis między wydajnością a zużyciem pamięci operacyjnej
    \end{itemize}

    \item Wpływ współczynnika $\alpha$ (wypełnienie strony):
    \begin{itemize}
        \item Wartości w przedziale 0.5 - 0.75 okazały się optymalne
        \item Niższe wartości $\alpha$ skutkują większym rozmiarem plików
        \item Wyższe wartości $\alpha$ zwiększają częstotliwość reorganizacji i liczbę operacji przy wstawianiu
    \end{itemize}

    \item Wpływ współczynnika $\beta$ (rozmiar obszaru przepełnień):
    \begin{itemize}
        \item Silnie wpływa na całkowite zużycie pamięci
        \item Mniejsze wartości $\beta$ są zalecane dla optymalizacji przestrzeni
        \item Zbyt niski $\beta$ może prowadzić do częstszych reorganizacji
    \end{itemize}

    \item Wpływ współczynnika $\gamma$ (próg reorganizacji):
    \begin{itemize}
        \item Niższe wartości $\gamma$ prowadzą do częstszych reorganizacji, ale utrzymują strukturę w lepszej kondycji
        \item Wyższe wartości $\gamma$ zmniejszają częstotliwość reorganizacji, ale mogą prowadzić do degradacji wydajności wyszukiwania
        \item Optymalny $\gamma$ zależy od proporcji operacji odczytu do zapisu
    \end{itemize}

    \item Ograniczenia metody:
    \begin{itemize}
        \item Koszt reorganizacji może być znaczący dla dużych zbiorów danych
        \item Wymaga odpowiedniego dostrojenia parametrów do konkretnego przypadku użycia
    \end{itemize}
\end{itemize}

\end{document}