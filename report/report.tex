\documentclass[12pt]{article}
\usepackage[utf8]{inputenc}
\usepackage{polski}
\usepackage[polish]{babel}
\usepackage{graphicx}
\usepackage{listings}
\usepackage{amsmath}
\usepackage{hyperref}

\title{Implementacja i analiza indeksowo-sekwencyjnej organizacji plików}
\author{Wojciech Trapkowski}
\date{\today}

\begin{document}
\maketitle

\section{Wprowadzenie}
Celem projektu jest implementacja i analiza indeksowo-sekwencyjnej organizacji plików (ISAM - Indexed Sequential Access Method), która łączy zalety dostępu sekwencyjnego i bezpośredniego do danych. Metoda ta została opracowana przez IBM w latach 60-tych i do dziś stanowi podstawę wielu systemów bazodanowych.

Podstawowe założenia tej organizacji plików obejmują:
\begin{itemize}
    \item Przechowywanie danych w uporządkowanej sekwencji rekordów, pogrupowanych w strony o stałym rozmiarze
    \item Utrzymywanie oddzielnego pliku indeksowego, zawierającego klucze i wskaźniki do odpowiadających im stron w pliku głównym
    \item Wykorzystanie obszaru przepełnień do obsługi nowych rekordów, których nie można umieścić w pierwotnie przydzielonych stronach
    \item Okresową reorganizację pliku w celu optymalizacji jego struktury
\end{itemize}

Organizacja ISAM zapewnia wydajne operacje wyszukiwania dzięki indeksom, zachowując jednocześnie możliwość sekwencyjnego przetwarzania danych. Jest szczególnie efektywna w systemach, gdzie stosunek operacji odczytu do zapisu jest wysoki, a dane są względnie statyczne.

\section{Struktury plików}
\begin{itemize}    
    \item Struktura pliku indeksowego zawiera strony indeksowe, gdzie każdy wpis składa się z klucza początkowego oraz wskaźnika do odpowiedniej strony w pliku głównym.

    \item Struktura pliku głównego składa się z nagłówka zawierającego liczbę stron oraz wskaźnik do obszaru przepełnień (strażnika), a następnie sekwencji stron zawierających rekordy.
    
    \item Obszar przepełnień służy do przechowywania rekordów, które nie mogą być umieszczone w pierwotnie przydzielonych stronach. Każdy rekord w obszarze głównym może wskazywać na dodatkowe rekordy w obszarze przepełnień.
    
    \item Organizacja rekordów w stronach opiera się na strukturze Page, która zawiera stałą liczbę wpisów. Każdy wpis zawiera klucz, wartość (PESEL), wskaźnik do obszaru przepełnień oraz flagę usunięcia.
\end{itemize}

\section{Szczegóły implementacyjne}
\subsection{Buforowanie w pamięci operacyjnej}
\begin{itemize}
    \item Mechanizm buforowania zaimplementowany jest w klasie PageBuffer, która wykorzystuje inteligentne wskaźniki do zarządzania stronami w pamięci.
    
    \item Wielkość bufora jest określona przez stałą, która definiuje maksymalną liczbę stron przechowywanych jednocześnie w pamięci.
    
    \item Strategia zastępowania stron opiera się na liczbie referencji do strony - usuwane są strony z pojedynczą referencją. Przed usunięciem strony z bufora, jej zawartość jest zapisywana na dysk.
    
    \item System śledzi liczbę operacji odczytu i zapisu poprzez liczniki.
\end{itemize}

\subsection{Parametry implementacyjne}
\begin{itemize}
    \item Rozmiar strony (PAGE\_SIZE) jest stałą określającą liczbę rekordów w pojedynczej stronie.
    
    \item Współczynnik wypełnienia ($\alpha$) określa maksymalną liczbę rekordów w stronie po reorganizacji.
    
    \item Współczynnik obszaru przepełnień ($\beta$) definiuje stosunek rozmiaru obszaru przepełnień do obszaru głównego
    
    \item Reorganizacja jest wykonywana gdy liczba rekordów w obszarze przepełnień przekroczy ustalony próg ($\gamma$).
\end{itemize}

\section{Format pliku testowego}
\subsection{Struktura rekordu}
W implementacji rekord jest reprezentowany jako pojedyncza liczba całkowita typu uint64\_t, przechowująca numer PESEL.

\section{Prezentacja wyników}
\subsection{Interfejs użytkownika}
Program oferuje interaktywny interfejs wiersza poleceń oraz możliwość wykonywania komend z pliku. Dostępne są następujące tryby pracy:
\begin{itemize}
    \item Tryb interaktywny - oznaczony znakiem zachęty ">"
    \item Tryb wsadowy - wykonywanie komend z pliku
\end{itemize}

\subsection{Dostępne komendy}
Program obsługuje następujące polecenia:
\begin{itemize}
    \item \texttt{insert <klucz> <wartość>} - wstawia nowy rekord
    \item \texttt{update <klucz> <wartość>} - aktualizuje istniejący rekord
    \item \texttt{search <klucz>} - wyszukuje rekord o podanym kluczu
    \item \texttt{remove <klucz>} - usuwa rekord o podanym kluczu
    \item \texttt{print} - wyświetla zawartość całej bazy danych
    \item \texttt{print\_stats} - wyświetla statystyki (liczba operacji I/O)
    \item \texttt{generate <liczba\_kluczy>} - generuje zadaną liczbę losowych rekordów
    \item \texttt{reorganise} - wymusza reorganizację struktury
    \item \texttt{flush} - wymusza zapis buforowanych danych na dysk
    \item \texttt{help} - wyświetla listę dostępnych komend
    \item \texttt{exit}/\texttt{quit} - kończy działanie programu
\end{itemize}

\subsection{Format wyświetlania}
\begin{itemize}
    \item Wyniki wyszukiwania są wyświetlane w formacie: wartość rekordu lub komunikat "Not found" dla nieznalezionych kluczy
    \item Błędy operacji są sygnalizowane odpowiednimi komunikatami
    \item Statystyki pokazują liczbę operacji odczytu i zapisu wykonanych na dysku
\end{itemize}

\section{Eksperymenty}
\subsection{Metodologia}
\begin{itemize}
    \item Opis przeprowadzonych testów
    \item Badane parametry
    \item Mierzone metryki
\end{itemize}

\subsection{Wyniki}
\subsubsection{Wpływ rozmiaru strony}
[Tu umieść wykresy i analizę wpływu rozmiaru strony]

\subsubsection{Wpływ współczynnika $\alpha$}
[Tu umieść wykresy i analizę wpływu współczynnika wypełnienia]

\subsubsection{Wpływ współczynnika $\beta$}
[Tu umieść wykresy i analizę wpływu rozmiaru obszaru przepełnień]

\subsection{Analiza złożoności operacji}
\begin{itemize}
    \item Wstawianie
    \item Wyszukiwanie
    \item Usuwanie
    \item Reorganizacja
\end{itemize}

\section{Wnioski}
\begin{itemize}
    \item Optymalne wartości parametrów
    \item Zalecane zastosowania
    \item Ograniczenia metody
\end{itemize}

\section{Bibliografia}
\begin{thebibliography}{9}
    % Tu dodaj źródła
\end{thebibliography}

\end{document}